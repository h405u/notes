\chapter{Undecidability}
\setcounter{section}{-1}

\section{Number Theory}

\textit{The language of number theory} is a first-order language with equality and parameters $\forall, 0, \mathrm{S}, <, +, \cdot$ and E, where E is a two-place function symbol. The intended structure for it is $\mathfrak{R}=(\mathbb{N};0,S,<,+,\cdot,E)$, where $E$ is exponentiation on $\mathbb{N}$. Thus \textit{number theory} is $\mathrm{Th}\mathfrak{R}$.

\begin{reference}{Thm}{t30a}
  Let $A\subseteq \mathrm{Th}\mathfrak{R}$, and assume that $\{\sharp\alpha|\alpha\in A\}$ is a set definable in $\mathfrak{R}$. Then we can find a sentence $\sigma$ such that $\sigma\in \mathrm{Th}\mathfrak{R}$ but $\sigma$ is not deducible from $A$.
\end{reference}
