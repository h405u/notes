\chapter{Finite-Dimensional Vector Spaces}

\section{Span and Linear Independence}

We will usually write lists of vectors without surrounding parentheses.

\begin{reference}{Defn}{linearcombination}
  A \textit{linear combination} of a list $v_1,\dots,v_m$ of vectors in $V$ is a vector of the form
  \[
    a_1v_1+\cdots+a_mv_m,
  \]
  where $a_1,\dots,a_m\in \mathbb F$.
\end{reference}

\begin{reference}{Defn}{span}
  The set of all linear combinations of a list of vectors $v_1,\dots,v_m$ in $V$ is called the \textit{span} of $v_1,\dots,v_m$, denoted $\mathrm{span}(v_1,\dots,v_m)$. The span of the empty list $()$ is defined to be $\{0\}$.
\end{reference}

\begin{reference}{Thm}{spanminimality}
  The span of a list of vectors in $V$ is the smallest subspace of $V$ containing all vectors in the list.
\end{reference}

The proof is omitted as trivial.

\begin{reference}{Defn}{spans}
  If $\mathrm{span}(v_1,\dots,v_m)$ equals $V$, say that the list $v_1,\dots,v_m$ \textit{spans} $V$.
\end{reference}

\begin{reference}{Defn}{finitedimensionalvectorspace}
  A vector space is called \textit{finite-dimensional} if some list of vectors in it spans the space.
\end{reference}
