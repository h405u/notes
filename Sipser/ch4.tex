\chapter{Decidability}

\section{Decidable Languages}

\begin{reference}{Thm}{adfa}
  (4.1) $A_{\mathrm{DFA}}=\{\langle B, w\rangle|B\text{ is a DFA that accepts }w\}$ is decidable.
\end{reference}

\begin{proof}[Proof Idea]
  We use a TM to simulate $B$ on $w$, this is clearly a decider.
\end{proof}

\begin{reference}{Thm}{anfa}
  (4.2) Also, $A_{\mathrm{NFA}}$ is decidable.
\end{reference}

\begin{proof}[Proof Idea]
  Convert the NFA to a DFA and use the TM from \ref{adfa} to decide. Actually an NFA might loop for that it takes $\varepsilon$. It actually still will be decidable, but that is something we have to prove.
\end{proof}

\begin{reference}{Thm}{edfa}
  (4.3) $E_{\mathrm{DFA}}=\{\langle A\rangle|A\text{ is a DFA and }L(A)=\emptyset\}$ is decidable.
\end{reference}

\begin{proof}[Proof Idea]
  We use BFS to see which of the states of $A$ are reachable.
\end{proof}

\begin{reference}{Thm}{eqdfa}
  (4.5) $EQ_{\mathrm{DFA}}=\{\langle A,B\rangle|A\text{ and }B\text{ are DFAs and }L(A)=L(B)\}$ is decidable.
\end{reference}

\begin{proof}[Proof Idea]
  Make DFA $C$ that accepts $w$ where $A$ and $B$ disagree (easily, $L(C)=(L(A)\cap \overline{L(B)})\cup(\overline{L(A)}\cap L(B))$) and test if $L(C)=\emptyset$. If we are to prove it by feeding strings into the simulated DFAs it would be way more complicated. A basic idea is to try to prove that if $L(A)\neq L(B)$, some strings whose length are at most some bound (say the sum or product of the numbers of states in $A$ and $B$) are going to behave differently when fed to $A$ and $B$.
\end{proof}

\begin{reference}{Thm}{acfg}
  (4.7) $A_{\mathrm{CFG}}=\{\langle G, w\rangle|G\text{ is a CFG that generates }w\}$ is decidable.
\end{reference}

\begin{proof}[Proof Idea]
  By \ref{cnfthm} and \ref{2.26} this is trivial.
\end{proof}

\textit{Comment.} So $A_{\mathrm{PDA}}$ is also decidable. Note that this is not easy to prove on its own, since PDAs may not halt.

\begin{reference}{Thm}{ecfg}
  (4.8) $E_{\mathrm{CFG}}=\{\langle G\rangle|G\text{ is a CFG and }L(G)=\emptyset\}$ is decidable.
\end{reference}

\begin{proof}[Proof Idea]
  Mark all the terminals in $G$. Then repeat the following until new variables are marked: mark all occurrences of variable $A$ if $A\to B_1B_2\cdots B_k$ is a rule and all $B_i$'s were already marked.
\end{proof}

\begin{reference}{Thm}{cfldeci}
  (4.9) Every context-free language is decidable.
\end{reference}

\begin{proof}[Proof Idea]
  It follows immediately from \ref{acfg}. Note that we know it is decidable, without knowing \textit{how} to decide it.
\end{proof}

\section{Undecidability}

\begin{reference}{Thm}{runc}
  $\mathbb{R}$ is uncountable.
\end{reference}

\begin{proof}
  We prove by contradiction and via diagonization. Say that each real number is assigned an index $i\in \mathbb{N}$. Then $r\in \mathbb{R}$ in its decimal representation that differs from the $n$th number in the $n$th digit for any $n\in \mathbb{N}$ is not $i$th number for any $i$. Hence $r\notin \mathbb{R}$ and contradiction.
\end{proof}

\begin{reference}{Cor}{lunc}
  The set of all languages over $\Sigma$ is not countable.
\end{reference}

\begin{proof}
  $\Sigma^*$ is countable. We can assign each language an infinite \textit{binary} decimal, with each digit of it representing if a string is present in the language.
\end{proof}

Observe that the set $\mathcal{M}$ of all turing machines is countable. So some languages are not recognizable, a fortiori decidable.

\begin{reference}{Note}{1stproblem}
  David Hilbert's first problem asked if there is a set of intermediate size between $\mathbb{N}$ and $\mathbb{R}$. Gödel and Cohen showed that we cannot answer this question by using the standard axioms of mathematics. We are even not sure if ``infinite sets'' has mathematical ``reality''.
\end{reference}

\begin{reference}{Rmk}{atmreco}
  $A_{\mathrm{TM}}$ is recognizable.
\end{reference}

We can prove it by simulating. Note that we do not ask the simulator TM to reject if $M$ rejects by looping (one should prove that this can be carried out effectively, but \ref{halt} shows that it cannot, for that we can not tell if a TM is looping on some input) and instead we say nothing about the looping condition, for that if $M$ rejects by looping, our simulator will as well. This simulator is of great historical importance, as the \textit{first} machine described (due to Alan Turing) to operate based on a stored program, and it later came to be known as the Von Neumann architecture.

\begin{reference}{Thm}{atm}
  (4.11) $A_{\mathrm{TM}}=\{\langle M,w\rangle|M\text{ is a TM that accepts }w\}$ is undecidable.
\end{reference}

\begin{proof}
  Assume some TM $H$ decides $A_{\mathrm{TM}}$. We use $H$ to construct TM
  \begin{align*}
    D=\text{``} & \text{On input }\langle M\rangle                                        \\
                & \text{1. Simulate }H\text{ on input }\langle M,\langle M\rangle\rangle. \\
                & \text{2. Accept if }H\text{ rejects and reject if }H\text{ accepts."}
  \end{align*}
  Then $D$ accepts $\langle D\rangle$ iff $D$ does not accept $\langle D\rangle$. Hence contradiction.
\end{proof}

\textit{Comment.} Surprisingly $\langle M,\langle M\rangle\rangle$ is of some real world applications, for example self-hosting languages whose compilers can be written in themselves. Check the following figure to see that this proof uses \textit{diagonalization}. To employ it we essentially attempt to show contradiction by constructing some item (by countability) that differs from all items of its type by some \textit{decidable} process.
\begin{figure}[H]
  \[
    \begin{array}{c|cccccc}
      \text{TMs} & \langle M_1 \rangle         & \langle M_2 \rangle         & \langle M_3 \rangle         & \langle M_4 \rangle         & \cdots & \langle D \rangle   \\
      \hline
      M_1        & \textcolor{red}{\text{acc}} & \text{rej}                  & \text{acc}                  & \text{acc}                  & \cdots & \text{acc}          \\
      M_2        & \text{rej}                  & \textcolor{red}{\text{rej}} & \text{rej}                  & \text{rej}                  & \cdots & \text{rej}          \\
      M_3        & \text{acc}                  & \text{acc}                  & \textcolor{red}{\text{acc}} & \text{acc}                  & \cdots & \text{acc}          \\
      M_4        & \text{rej}                  & \text{rej}                  & \text{acc}                  & \textcolor{red}{\text{acc}} & \cdots & \text{rej}          \\
      \vdots     & \vdots                      & \vdots                      & \vdots                      & \vdots                      & \ddots & \vdots              \\
      D          & \textcolor{red}{\text{rej}} & \textcolor{red}{\text{acc}} & \textcolor{red}{\text{rej}} & \textcolor{red}{\text{rej}} & \cdots & \textcolor{blue}{?}
    \end{array}
  \]
  \caption{Undecidability hinges on diagonalization.}
\end{figure}

% TODO: elaborate on self-hosting languages, especially in its merits and implications (sufficient expressiveness, etc.) Better state the commonalities between this example and R being uncountable: what does the 'uncountability' of R correspond to in this proof? Show that we cannot use the enumeration of all strings (instead of $\langle M_i\rangle$) for losing decidability (of some sort).

\begin{reference}{Thm}{corecodec}
  A language is decidable iff both it and its complement are Turing-recognizable.
\end{reference}

\begin{proof}
  We construct a decider by simulating both TMs in parallel (alternately).
\end{proof}

\begin{reference}{Cor}{unreco}
  $\overline{A_{\mathrm{TM}}}$ is unrecognizable.
\end{reference}

\textit{Comment.} \ref{corecodec} can be stated more generally: Let $D$ and $L$ be two languages, and $L$ a decidable one. Then $D\cap L$ is decidable iff both it and $\overline{D}\cap L$ is recognizable.

\section*{Exercises and Problems}

\setcounter{exercise}{13}

\begin{exercise}{4.14}
  Show that
  \[
    \{\langle G\rangle|G\text{ is a CFG over }\{0,1\}\text{ and }1^*\cap L(G)\neq\emptyset\}
  \]
  is decidable.
\end{exercise}

\textcolor{red}{to do (hard!)}

\setcounter{exercise}{17}

\begin{exercise}{4.18}
  Let $C$ be a language. Prove that $C$ is T-recognizable iff a decidable language $D$ exists such that $C=\{x|\exists y(\langle x,y\rangle\in D)\}$.
\end{exercise}

\begin{proof}
  ($\Leftarrow$) Consider a string $x$, we enumerate $y\in \Sigma^*$ and see if $\langle x,y_i\rangle\in D$ for at least one $i$. Thus $C$ is recognizable. *($\Rightarrow$) Say that $L(M)=C$, where $M$ is a TM. Let $D=\{\langle x,y\rangle|M\text{ accepts }x\text{ in }y\text{ steps}\}$, and we are done.
\end{proof}

\textit{Comment.} A bound that can approach infinity can be useful to turn something recognizable into something decidable.

\setcounter{exercise}{23}

\begin{exercise}{4.24}
  A \textit{useless state} in a pushdown automaton is never entered on any input string. Consider the problem of determining whether a pushdown automaton has any useless states. Formulate this problem as a language and show that it is decidable.
\end{exercise}

\textcolor{red}{to do}

\setcounter{exercise}{26}

\begin{exercise}{4.27}
  Show that
  \[
    E=\{\langle M\rangle|M\text{ is a DFA that accepts some string with more 1s than 0s}\}
  \] is decidable.
\end{exercise}

\textcolor{red}{to do}
